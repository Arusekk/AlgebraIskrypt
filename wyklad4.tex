\documentclass[12pt,a4paper]{article}

\usepackage{amsfonts}
\usepackage{amssymb}
\usepackage{amsmath}
\usepackage{amsthm}
\usepackage{enumerate}

\usepackage[utf8]{inputenc}
\usepackage{csquotes}
\usepackage{polski}
\usepackage[utf8]{inputenc} 
\usepackage[T1]{fontenc}
%\usepackage{indentfirst}
\usepackage{ wasysym }
\usepackage{hyperref}

\usepackage{tikz}
\usetikzlibrary{arrows}

\frenchspacing

%\renewcommand{\arraystretch}{3.0}
%\renewcommand{\thefootnote}{\fnsymbol{footnote}}


\newcommand{\norm}[1]{\left\lVert#1\right\rVert}

% Pisanie za każdym razem \mathbb{R} jest zbyt długie, teraz wystarczy \RR.
\newcommand{\RR}{\mathbb{R}}
% tak samo \overrightarrow
\newcommand{\V}[1]{\overrightarrow{#1}}
\newcommand{\Pair}[2]{\left<#1,#2\right>}
\renewcommand{\qed}{$\square$}

% Definicja stylów:
\theoremstyle{plain}
\newtheorem{tw}{Twierdzenie}[section]
\theoremstyle{definition}
\newtheorem{ft}{Fakt}[section]
\theoremstyle{definition}
\newtheorem{df}{Definicja}[section]
\theoremstyle{definition}
\newtheorem*{nt}{Notacja}
\theoremstyle{definition}
\newtheorem*{dd}{Dowód}
\theoremstyle{definition}
\newtheorem*{lem}{Lemat}
\theoremstyle{definition}
\newtheorem*{prz}{Przykład}
\theoremstyle{definition}
\newtheorem*{przy}{Przykłady}
\theoremstyle{definition}
\newtheorem*{uw}{Uwaga}
\theoremstyle{definition}
\newtheorem*{wn}{Wnioski}

\begin{document}
\begin{tw} 
    ~\\
    $ F: V \rightarrow W \text{ p. liniowe}$ \\
    baza $V$ --- $B=(b_1, ..., b_n)$ \\
    baza $W$ --- $C=(c_1, ..., c_n)$ 
    $$ [F(v) ]_C = m^B_C(F)[v]_B $$ 
    gdzie $m^B_C(F)=([F(b_1)]_C, ..., [F(b_n)]_C)$.
\end{tw}

\begin{dd}
    Weźmy $v \in V$ i~jego przedstawienie $v=\sum\limits_{i=1}^n\alpha_ib_i = [v]_B$.
    \[F(v)=F\left(\sum_{i=1}^n\alpha_ib_i\right)=\sum_{i=1}^n\alpha_iF(b_i)\]
    \[F(b_i)=\sum_{j=1}^m\beta_{ji}c_j\]
    gdzie
    \[\begin{pmatrix}\beta_{1i} \\ \vdots \\ \beta_{mi}\end{pmatrix}=[F(b_i)]_C\]
    \[F(v)=\sum_{i=1}^n\alpha_iF(b_i)=\sum_{i=1}^n\alpha_i\sum_{j=1}^m\beta_{ji}c_j=\]
    \[=\sum_{i=1}^n\sum_{j=1}^m(\alpha_i\beta_{ji})c_j=\sum_{j=1}^m\left(\sum_{i=1}^n\alpha_i\beta_{ji}\right)c_j\]
    \[[F(v)]_C=\begin{pmatrix}\sum\limits_{i=1}^n\beta_{1i}\alpha_i\\ \vdots \\\sum\limits_{i=1}^n\beta_{mi}\alpha_i\end{pmatrix}=
     \begin{pmatrix}\beta_{11}& \dots &\beta_{1n}\\ \vdots & \vdots & \vdots\\\beta_{m1}& \dots&\beta_{mn}\end{pmatrix}
    \begin{pmatrix}\alpha_1\\ \vdots \\\alpha_n\end{pmatrix}  \] \hfill \qed
\end{dd}

\begin{wn}
    \[ B = (b_1, ..., b_n) \]
    \[ C = (c_1, ..., c_n) \]
    obie są bazami $V$, więc
    \[ [V]_C = m^B_C(id)[v]_B \]
    gdzie $m^B_C(id)=([b_1]_C, ..., [b_n]_C)$
    
    Jest to macierz przekształcenia.
    Baza standardowa $\varepsilon=(e_1, ..., e_n)$ występuje
    w~przestrzeniach $K^n$, gdzie
    \[e_i=\begin{pmatrix}0 \\ \vdots \\ 0 \\ 1 \\ 0 \\ \vdots \\ 0\end{pmatrix} \leftarrow i\]
\end{wn}

\begin{tw}
    \[ F: V \rightarrow W \mathrm{p. liniowe} \]
    \[ G: U \rightarrow V \mathrm{p. liniowe} \]
    baza $U$ --- $B=(b_1, ..., b_k)$ \\
    baza $V$ --- $C=(c_1, ..., c_l)$ \\
    baza $W$ --- $D=(d_1, ..., d_m)$ \\
    \[m^B_D(F\circ G)=m^C_D(F)m^B_C(G)\]
\end{tw}

\begin{dd} 
    ~\\
    $U \overset{G}{\longrightarrow}  V \overset{F}{\longrightarrow} W$ \\ 
    $[(F\circ G)(u)]_D = m_D^B(F\circ G)[u]_B$ \\
    $[F(G(u))]_D = m_D^C(F) \cdot [G(u)]_C = m_D^C \cdot (m_C^B(G)[v]_B)$
\end{dd}

\begin{wn} \hfill
    \begin{itemize}
        \item $K^k \overset{G}{\longrightarrow} K^l \overset{F}{\longrightarrow} K^m $\\
              $ F(X) = AX, \qquad G(X) = BX $ \\ 
              $(F\circ G)(X) = F(G(X)) = A(BX) = (AB)X$
        \item $ F: V \rightarrow W \mathrm{ p.liniowe} $ \\
              $ F^{-1}: W \rightarrow V \mathrm{ p. liniowe} $ \\
              baza $V$ --- $B=(b_1, ..., b_n)$ \\
              baza $W$ --- $C=(c_1, ..., c_n)$ \\
              $m^B_C(F)m^C_B(F^{-1}) = I $ \\
              $m^C_B(F^{-1})m^B_C(F) = I $
    \end{itemize}
\end{wn}

\begin{df}
    Przekształcenie liniowe $F: V \rightarrow W$ nazywamy
    \textbf{izomorfizmem}, jeśli istnieje przekształcenie
    liniowe $G: W\rightarrow V$, takie że $G\circ F = id_V$
    i~$F\circ G=id_W$.
\end{df}

\begin{ft}
    $F: V \rightarrow W$ jest izomorfizmem $\Leftrightarrow$ jest liniową
  bijekcją.
\end{ft}

\begin{dd} ~\\
    $\Rightarrow$ oczywiste\\
  $\Leftarrow$ $F^{-1}: W \rightarrow V$ (funkcja)
  Potrzeba pokazać, że $F^{-1}$ jest liniowa. Przykładowo:
  $$F^{-1}(w+w')=F^{-1}(F(v)+F(v'))=F^{-1}(F(v+v'))=v+v'=F^{-1}(w)+F^{-1}(w')$$ 
  Analogicznie dowodzimy jednorodność.
\end{dd}

\begin{df}
    Przestrzenie liniowe $V$ i~$W$ są izomorficzne ($V \simeq W$), gdy istnieje
    izomorfizm $F: V \rightarrow W$.
\end{df}

\begin{przy}
    \[K^n \simeq K_{n-1}[x]\]
    \[F: K^n \rightarrow K_{n-1}[x]\]
    \[\begin{pmatrix}a_0\\...\\a_n\end{pmatrix} \overset{F}{\longmapsto}
    \left(a_0+a_1x+...+a_{n-1}x^{n-1}\right)\]
    
    $F$ --- bijekcja, liniowa.
    
    \vspace{5mm}
    $M_{m\times n}(K)$ --- przestrzeń liniowa nad $K$.
  \[M_{m\times n}(K) \simeq K^{mn}\]
  \[F: M_{m\times n}(K) \rightarrow K^{mn}\]

  \[(a_{ij}) \longmapsto \begin{pmatrix}a_{11}\\...\\a_{1n}\\
    a_{21}\\...\\a_{mn}\end{pmatrix}\]
\end{przy}

\begin{tw}
    Jeśli $V$ jest skończenie wymiarową przestrzenią liniową nad $K$, to
  \[V \simeq K^n\]
  gdzie $n=\dim V$.
\end{tw}

\begin{dd}
    Niech $B=(b_1, ..., b_n)$ będzie bazą $V$. Wówczas
  \[F: V \rightarrow K^n\]
  \[F(v) = [v]_B\]
  jest izomorfizmem.
\end{dd}

\section{Macierz odwrotna}

\begin{df}
    \textbf{Macierzą odwrotną} do $A \in M_{n\times n}(K)$ nazywamy macierz
  $B \in M_{n\times n}(K)$ taką, że
  \[AB = BA = I\]
\end{df}

\begin{ft}
    Dla $A,B \in M_{n\times n}(K)$ zachodzi
    \[AB=I \Rightarrow BA=I\]
    Dowód: ćwiczenie.
\end{ft}

\begin{df}
    Operacjami elementarnymi (wierszowymi) na macierzy prostokątnej $A$
    nazywamy następujące operacje:
    \begin{enumerate}[{(}1{)}]
        \item zamiana miejscami $i$-tego i~$j$-tego wiersza ($i \neq j$)
        \item przemnożenie $i$-tego wiersza przez $\alpha \in K \setminus \{0\}$
        \item dodanie do $j$-tego wiersza $\alpha$-krotności $i$-tego wiersza
          ($i \neq j$, $\alpha \in K \setminus \{0\}$)
    \end{enumerate}
    Analogicznie definiujemy operacje elementarne (kolumnowe).
\end{df}

\begin{uw}
    Wszystkie te operacje można zapisać mnożąc z~lewej (wierszowe) lub
  z~prawej (kolumnowe) przez odpowiednią macierz.

  Ad 1. (zamiana $i$ i $j$)
  \[\begin{pmatrix}
     1 &...&...&...\\
    ...& 0 & 1 &...\\
    ...& 1 & 0 &...\\
    ...&...&...& 1
  \end{pmatrix}\begin{pmatrix}
    a_{11}&...&a_{1n}\\
    ...&a_{jk}&...\\
    ...&a_{ik}&...\\
    a_{m1}&...&a_{mn}
  \end{pmatrix}=\begin{pmatrix}
    a_{11}&...&a_{1n}\\
    ...&a_{ik}&...\\
    ...&a_{jk}&...\\
    a_{m1}&...&a_{mn}
  \end{pmatrix}\]

  Analogicznie, wykonanie elementarnej operacji na $A$ to
  mnożenie $A$ przez pewną macierz $B_{op}$.
  
  $B_{op}$ to po prostu $I$, na ktorym została wykonana ta operacja.
\end{uw}

\begin{ft}
    \[A = (a_{ij}) \in M_{m\times n}(K)\]
  \begin{enumerate}
    \item Jeśli ciąg operacji elementarnych \textit{wierszowych} przeprowadza
      macierz \[(A|I)=\begin{pmatrix}
        a_{11}&...&a_{1n}& 1 &...& 0 \\
        ...&...&...&...& 1 &...\\
        a_{m1}&...&a_{mn}& 0 &...& 1 \\
      \end{pmatrix}\]
      na $(I|B)$, to $B = A^{-1}$
    \item Albo istnieje ciąg operacji elementarnych wierszowych
      przeprowadzających $A$ na~$I$, albo istnieje ciąg operacji
      wierszowych zamieniających $A$ na~macierz z~zerową kolumną /
      zerowym wierszem
  \end{enumerate}
\end{ft}

\begin{dd}
    \[(E_k...E_2E_1)A=I\]
  \[E_k...E_2E_1I=E_k...E_2E_1=A^{-1}\]
\end{dd}

\section{Wyznacznik}
    Intuicyjnie: $n$-wymiarowa objętość (dim = 1: długość, dim = 2: pole powierzchni, dim = 3: objętość). \\
    Zerowy wyznacznik oznacza liniową zależność wektorów w~macierzy. \\
    Wzory Kramera --- rozwiązywanie układów równań przy pomocy wyznaczników.

\begin{df}
    Odwzorowanie
  $F: \underbrace{V \times V \times ... \times V}_{n} \rightarrow K$
  (gdzie $V$ jest przestrzenią liniową nad $K^n$) nazywamy:
  \begin{enumerate}
    \item \textbf{wieloliniowym}, jeśli
      $\forall_k\forall_{v_i,v_{i'}\in V}\forall_{\alpha\in K}$:
      \[F(v_1, ..., v_k+v_{k'}, ..., v_n)
      = F(v_1, ..., v_k+v_{k'})\]
      oraz
      \[F(v_1, ..., \alpha v_k, ..., v_n) = \alpha F(v_1, ..., v_n)\]
    \item \textbf{antysymetrycznym}, jeśli $\forall_{i,j}$:
      \[F(v_1, ..., v_i, ..., v_j, ..., v_n)
      = -F(v_1, ..., v_j, ..., v_i, ..., v_n)\]
  \end{enumerate}
\end{df}

\begin{tw}
    Dla każdej liczby $c \in K$ istnieje dokładnie jedno odwzorowanie
    \[F: \underbrace{K^n \times K^n \times ... \times K^n}_{n} \rightarrow K\]
    które jest $n$-liniowe, antysymetryczne oraz $F(I) = c$.
\end{tw}

\begin{dd}
    \[e_i=\begin{pmatrix}0\\ \vdots \\1\\ \vdots \\0\end{pmatrix} \leftarrow i\]
  \[A=(a_{ij})=(A_1, ..., A_n)\]
  \[F(A_1, ..., A_n) = F(\sum_{i_1}a_{i_1}e_{i_1}, ..., \sum_{i_n}a_{i_n}e_{i_n}) =\]
  (z~$n$-liniowości)
  \[= \sum_{i_1}...\sum_{i_n}a_{i_1}...a_{i_n}F(e_{i_1}, ..., e_{i_n}) =\]
  (z~antysymetryczności) $F(...v...v...)=-F(...v...v...)$,
  czyli $2F(...v...v...)=0$. Dla ciał, że $1+1=0$ też, z~inną def. antysymetryczności)
  \[= \sum_{\sigma\in S_n}a_{\sigma(1)1}...a_{\sigma(n)n}F(e_{\sigma(1)}, ..., e_{\sigma(n)}) = \]
  \[= \sum_{\sigma\in S_n} \pm a_{\sigma(1)1}...a_{\sigma(n)n}F(e_1, ..., e_n)\]
  gdzie $F(e_1, ..., e_n)=c$.
  \[\sigma: \{1, 2, ..., n\} \overset{\mathrm{na}}{\underset{\mathrm{,,1-1''}}{\longrightarrow}} \{1, 2, ..., n\}\]

\end{dd}

\begin{lem}
    Jeśli permutację $\sigma$ rozłożymy na złożenie transpozycji, to $(-1)^{\#\mathrm{transpozycji}}$ jest niezależna od rozkładu (i~oznaczana jest $\mathrm{sgn}(\sigma)$).
\end{lem}

\begin{dd}
    Zdefiniujmy $t(\sigma)$ jako:
  \[t(\sigma)=\{(i,j): i<j \land \sigma(i)>\sigma(j)\}\]
  liczbę nieporządków\\
  Przykład:
  \[\begin{pmatrix}1&...&n\\\sigma(1)&...&\sigma(n)\end{pmatrix}
    =\begin{pmatrix}1&2&3&4&5\\5&4&1&3&2\end{pmatrix}\]

  \[t(\sigma) = 8\]
  \[\mathrm{sgn}(\sigma) = (-1)^8 = +1\]
  \[\sigma(id) = 0\]

  Transpozycja to permutacja postaci:
  \[\begin{pmatrix}
    1&...&i&...&j&...&n\\
    1&...&j&...&i&...&n
  \end{pmatrix}\]

  \[t(\mathrm{transpozycja})=2(j-i-1)+1\]
  a~zatem
  \[\mathrm{sgn}(\mathrm{transpozycja})=-1\]

  Wystarczy pokazać, że jeśli
  \[\sigma' = \tau \circ \sigma\;,\qquad \tau=\mathrm{transpozycja}\]
  to
  \[t(\sigma') \equiv t(\sigma)+1 \;(\mathrm{mod}\; 2)\]
  \[1^\circ\; \sigma(i) < \sigma(j)\]
  \[2^\circ\; \sigma(i) > \sigma(j)\]
  czyli \#transpozycji $\equiv t(\sigma)$ (mod 2).
  zatem $(-1)^{\#\mathrm{transpozycji}}$ nie zależy od rozkładu.\\
  Pozostaje sprawdzić, że otrzymany wzór
  \[F(A_1, ..., A_n)=\sum_{\sigma \in S_n}\mathrm{sgn}(\sigma)a_{\sigma(1)1}...a_{\sigma(n)n}\]
  spełnia zadane warunki. \qed
\end{dd}

\begin{df}
    Wyznacznikiem nazywamy \textit{jedyną} funkcję
  \[\det: (K^n)^n \rightarrow K\]
  \[\left((K^n)^n \simeq M_{n \times n}(K)\right)\]
  która jest $n$-liniowa, antysymetryczna i~$\det I=1$
\end{df}

\begin{ft}
    \[\det A^T = \det A\]
  \[\begin{vmatrix}1&2&3\\1&4&1\\2&0&0\end{vmatrix}
    =\begin{vmatrix}1&1&2\\2&4&0\\3&1&0\end{vmatrix}\]
\end{ft}
\begin{dd}
    \[\sum_\sigma\mathrm{sgn}(\sigma)a_{\sigma(1)1}...a_{\sigma(n)n}\]
    \[\sum_\sigma\mathrm{sgn}(\sigma^{-1})a_{\sigma^{-1}(1)1}...a_{\sigma^{-1}(n)n}\]
    \[\sum_{\sigma^{-1}}\mathrm{sgn}(\sigma^{-1})a_{\sigma^{-1}(1)1}...a_{\sigma^{-1}(n)n}\]
\end{dd}
\newpage
\begin{wn}
    Dla macierzy $A \in M_{n \times n}(K)$ wyznacznik $\det A$:
    \begin{enumerate}[{(}1{)}]
        \item zmienia znak przy zamianie miejscami dwóch wierszy/kolumn
        \item nie zmienia się po dodaniu krotności jednego wiersza/kolumny do innego
        \item mnoży się przez $t \in K$ przy pomnożeniu przez $t$ wybranego
          wiersza/kolumny
    \end{enumerate}
\end{wn}

\begin{dd} \hfill
    \begin{enumerate}[{(}1{)}]
        \item Antysymetryczność
        \item $det(A_1,\dots,A_i+tA_j,\dots,A_n) = det(A_1,\dots,A_i,\dots,A_n) + \\ + t \overbrace{det(A_1,\dots,A_i,\dots,A_i
        ,dots,A_n)}^0 = det(A_1,\dots,A_n)$
        \item $det(A_1,\dots,tA_i,\dots,A_n) = t det(A_1,\dots,A_i,\dots,A_n)$
    \end{enumerate}
\end{dd}

\begin{tw}[o rozwinięciu Laplace'a] ~\\
    Dla $A \in M_{n \times n} (K) $ oraz ustalonego $i \in \{1,\dots,n\}$ 
        $$det(A) = \sum_{j=1}^n (-1)^{i+j} a_{ij} A_{ij}$$
        gdzie $A_{ij}$ to wyznacznik macierzy A z usuniętym i-tym wierszem i j-tą kolumną. 
        \end{tw}
\begin{uw}
    Twierdzenie jest prawdziwe gdy zamienimy miejscami i i j.
\end{uw}
\end{document}