\documentclass[12pt,a4paper]{article}

\usepackage{amsfonts}
\usepackage{amssymb}
\usepackage{amsmath}
\usepackage{amsthm}
\usepackage{enumerate}

\usepackage[utf8]{inputenc}
\usepackage{csquotes}
\usepackage{polski}
\usepackage[T1]{fontenc}
%\usepackage{indentfirst}
\usepackage{ wasysym }
\usepackage{hyperref}

\usepackage{tikz}
\usetikzlibrary{arrows}

\frenchspacing

%\renewcommand{\arraystretch}{3.0}
%\renewcommand{\thefootnote}{\fnsymbol{footnote}}


\newcommand{\norm}[1]{\left\lVert#1\right\rVert}

% Pisanie za każdym razem \mathbb{R} jest zbyt długie, teraz wystarczy \RR.
\newcommand{\RR}{\mathbb{R}}
% tak samo \overrightarrow
\newcommand{\V}[1]{\overrightarrow{#1}}
\newcommand{\Pair}[2]{\left<#1,#2\right>}
\newcommand{\emtarny}{\begin{pmatrix}0\\\vdots\\1\\\vdots\\0\end{pmatrix}}
\newcommand{\skreslona}{\begin{vmatrix}&|&\\-&+&-\\&|&\end{vmatrix}}
\renewcommand{\qed}{$\square$}

% Definicja stylów:
\theoremstyle{plain}
\newtheorem{tw}{Twierdzenie}[section]
\theoremstyle{definition}
\newtheorem{ft}{Fakt}[section]
\theoremstyle{definition}
\newtheorem{df}{Definicja}[section]
\theoremstyle{definition}
\newtheorem*{nt}{Notacja}
\theoremstyle{definition}
\newtheorem*{dd}{Dowód}
\theoremstyle{definition}
\newtheorem*{lem}{Lemat}
\theoremstyle{definition}
\newtheorem*{prz}{Przykład}
\theoremstyle{definition}
\newtheorem*{przy}{Przykłady}
\theoremstyle{definition}
\newtheorem*{uw}{Uwaga}
\theoremstyle{definition}
\newtheorem*{wn}{Wnioski}

\begin{document}
\begin{dd}
  (lematu)

  Jeśli $F$ jest odwracalna, to istnieje $G: K^n \rightarrow K^n$, $G(X)=BX$,
  takie że
  \[F \circ G = G \circ F = id\]
  \[m(F \circ G) = m(G \circ F) = I\]
  \[AB = BA = I\]
  czyli $B=A^{-1}$ ($A$ odwracalna).
  \[1 = \det (AB) \overset{\text{ćw}} = \det A \det B\]
  więc $\det A \neq 0$.

  W~drugą stronę ($\det A \neq 0 \Rightarrow A \text{ odwracalna} \land B \text{ odwracalna}$) to wniosek z~następnego faktu.
  \hfill \qed
\end{dd}
\begin{ft}
  $B = (b, j)$ jest macierzą odwrotną do $A$, wtedy i~tylko wtedy, gdy:
  \[b_{ji} = \frac1{\det A}(-1)^{i+j}\skreslona\]
\end{ft}
\begin{prz}
  \[A = \begin{pmatrix}1&0&1\\0&2&1\\1&1&1\end{pmatrix}\qquad\det A = -1\]
  \[A^{-1}=\frac1{-1}\begin{pmatrix}1&+1&-2\\+1&0&-1\\-2&-1&2\end{pmatrix}^T
    =\begin{pmatrix}-1&-1&2\\-1&0&-1\\2&1&-2\end{pmatrix}\]
\end{prz}
\begin{dd}
  (faktu)

  \[AB=I\]
  \[A\cdot(B_1, \dots, B_n)=(e_1, \dots, e_n)\]
  \[AB_i=e_i \qquad e_i = \emtarny i\]
  układ Cramera.
  \[A\begin{pmatrix}b_{1i}\\\vdots\\b_{ni}\end{pmatrix} = \emtarny i\]
    \[b_{ji}=\frac{\det(A_1,\dots,e_i,\dots,A_n}{\det A}=\frac1{\det A}i
    \begin{vmatrix}
      a_{11}&\dots&0&\dots&a_{1n}\\
      &&\vdots&&\\
      &&1&&\\
      &&\vdots&&\\
      a_{n1}&\dots&0&\dots&a_{nn}
    \end{vmatrix} = \frac1{\det A}1\cdot(-1)^{i+j}\skreslona\]
\end{dd}
\begin{uw}
  Operacje elementarne (wierszowe i~kolumnowe) nie wpływają na zerowanie się
  wyznacznika.
\end{uw}
\begin{ft}
  Dany jest \textit{jednorodny} układ równań liniowych ($a_{ij} \in K$).
  \[\text{(J)}\left\{\begin{array}{l}
      a_{11}x_1+\dots+a_{1n}x_n = 0
      \\\vdots\\
      a_{m1}x_1+\dots+a_{mn}x_n = 0
    \end{array}\right.\]
  Zbiór rozwiązań układu (J) jest podprzestrzenią $K^n$.
\end{ft}
\begin{dd}
  Oznaczmy $X = \begin{pmatrix}x_1\\\vdots\\x_n\end{pmatrix}$.
  \[F: K^n \rightarrow K^m, F(X)=AX\]
  zbiór rozwiązań to $\mathrm{ker}F < K^n$.
  \hfill \qed
\end{dd}
\begin{ft}
  Dany jest \textit{niejednorodny} układ równań liniowych
  ($a_{ij}, b_i \in K$):
  \[\text{(NJ)}\left\{\begin{array}{l}
      a_{11}x_1+\dots+a_{1n}x_n = b_1
      \\\vdots\\
      a_{m1}x_1+\dots+a_{mn}x_n = b_m
    \end{array}\right.\]
  Zbiór rozwiązań układu (NJ) jest albo $\emptyset$ albo ${v}+\mathrm{ker}F$,
  gdzie:
  \begin{itemize}
    \item $F: K^n \rightarrow K^m, F(X)=AX$
    \item $v$ - \textit{dowolne} rozwiązanie (NJ).
  \end{itemize}
\end{ft}
\begin{dd}
  \[F^{-1}[b]\]

  $1^\circ$ $b \notin \mathrm{Im} F$
  \[F^{-1}[b] = \emptyset\]

  $2^\circ$ $b \in \mathrm{Im} F$ \\
  \begin{center}$b= F(v)$ dla pewnego (jakiegokolwiek) $v\in K^n$\end{center}

  $v'$ jest rozwiązaniem (NJ) $\Leftrightarrow F(v')=b$
  \[F(v+(v'-v)) = b\]
  \[F(v)+F(v'-v) = b\]
  \[b+\overbrace{F(v'-v)}^0 = b\]
  \[v'-v \in \mathrm{ker}F\]
  \hfill \qed
\end{dd}
\begin{def}
  \textit{Warstwą} podprzestrzeni $W < V$ nazywamy każdy zbiór postaci
  $v + W \subseteq V$. Zbiór wszystkich warstw podprzestrzeni $W < V$
  oznaczamy $V/W$.
\end{def}  
\begin{ft}
  Jeśli na~przestrzeni liniowej $V$ wprowadzimy relację
  \[\sim_W: v\sim_Wv' \Leftrightarrow v-v'\in W\]
  (gdzie $W$ --- ustalona podprzestrzeń), to:
  \begin{enumerate}
    \item $\sim_W$ jest relacją równoważności
    \item $[v]_{\sim_W} = v+W$
  \end{enumerate}
  Dowód --- ćw.
\end{ft}
\begin{ft}
  $V/W$ z~działaniami:
  \begin{itemize}
    \item $(v+W)+(v'+W) = (v+v')+W$
    \item $\alpha(v+W) = (\alpha v) + W$
  \end{itemize}
  jest przestrzenią liniową (ilorazową).
\end{ft}
\begin{ft}
  $\dim(V/W) = \dim V - \dim W$
\end{ft}
\begin{dd}
  \[V \ni v \overset F\longmapsto v+W \in V/W\]
  \[\mathrm{ker}F = W\]
  twierdzenie o indeksie:
  \[\dim V = \underbrace{\dim\mathrm{ker}F}_{\dim W} + \underbrace{\dim\mathrm{Im}F}_{\dim(V/W)}\]
  \hfill \qed
\end{dd}
\end{document}
